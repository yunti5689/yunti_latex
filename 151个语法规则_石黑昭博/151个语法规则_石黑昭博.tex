\documentclass{article}
\usepackage{fontspec}        % 对于 XeLaTeX 和 LuaLaTeX
\usepackage{xeCJK}          % 支持中文
\setmainfont{SimSun}        % 设置中文字体(选择合适的字体)
\setlength{\parindent}{2em} % 段落首行缩进

\title{151个一学就会的语法规则}
\author{石黑昭博}
\date{October 2024}

\begin{document}
\tableofcontents

\maketitle

\section{句子的构成}
\subsection{句子, 词, 短语和从句}
\subsubsection{英文的句子}
\subsubsection{词类和功能}
    The man said to me firmly and clearly, "Well, you must slove the two big problems."
\subsubsection{短语和从句}
\subsection{句子的要素}
\subsubsection{主部和述部, 主语和谓语}
\subsubsection{宾语}
\subsubsection{补足语}
\subsubsection{修饰语}
\section{句子的种类}
\subsection{英文的词序}
\subsection{句子的基本形式}
\subsubsection{陈述句}
\subsubsection{疑问句}
\subsubsection{祈使句}
\subsubsection{感叹句}
\subsection{疑问句, 祈使句的变化}
\section{动词与句型}
\subsection{动词的用法与句型}
page 38
\section{动词与时态}
\section{完成时}
\section{情态动词}
\section{语态}
\section{动词不定式}
\section{动名词}
\section{分词}
\section{比较}
\section{关系词}
\section{虚拟语气}
\section{疑问词与疑问句}
\section{否定}
\section{引语}
\section{名词化用法与无生命事务作主语}
\section{强调, 倒装, 插入, 省略, 同位语}
\section{名词}
\section{冠词}
\section{代词}
\section{形容词}
\section{副词}
\section{介词}
\section{连词}
\end{document}