\documentclass{article}
\usepackage{fontspec}        % 对于 XeLaTeX 和 LuaLaTeX
\usepackage{xeCJK}          % 支持中文
\setmainfont{SimSun}        % 设置中文字体(选择合适的字体)
\setlength{\parindent}{2em} % 段落首行缩进

\title{审美力}
\author{吴冠中}
\date{October 2024}

\begin{document}

\maketitle

\section{美之力}
\subsection{美盲要比文盲多}

行路长见闻。一路名胜之多,令人目不暇接,而“美盲”之多,亦是见闻之一。

我乘船去长江支流大宁河的小三峡游览,发现同舟的几对青年男女,每人手里一本小人书,撇开两岸的大好风光,看书度光阴。另见一胜地,陈列了许多老树根,神态突兀,确是极好的欣赏对象,然而也许正是为了“欣赏”的缘故吧,它们分别被涂上了各种颜色!

我赶到山西芮城看元代永乐宫的壁画,交通十分不便,一路打听时,常常听到一些熟悉当地的好心人的劝告:那里没有什么可玩的,很苦,你们那么大年纪,何必赶去!确实,看壁画的人并不多,显得冷冷清清。

我见过的寺庙不算少,近几年来又都香烟缭绕,拥挤的人群在顶礼膜拜菩萨。菩萨大都是被作为紧急任务赶塑起来的,因原先的早在“文革”中被革掉了命。新菩萨与老菩萨之间,实在已没有丝毫的血缘关系了,艺术的血缘啊!

一月奔波,最大的收获是饱看了南阳的汉画像石。南阳是刘秀的家乡,虽说帝皇本无种,南阳却因此布满了无数皇亲国戚的巨大陵墓。单就汉画馆里陈列的部分画像石看,其艺术的气概与魅力,已够令人惊心动魄了。那粗犷的手法,准确扼要的表现,把繁杂的生活情景与现实形态概括、升华成艺术形象,精微的细节被统一在大胆的几何形与强烈的节奏感中。其中许多关键的、基本的艺术法则与规律,正是从西方后期印象派开始所探寻的瑰宝!谁是汉画的作者?作者巨匠们很有可能是不识字的文盲,但通过实践与借鉴,却创造了伟大的艺术。文盲与美盲不是一回事,二者间不能画等号,识字的非文盲中倒往往有不少不分美丑的美盲!

那天正是清明节,成群的小学生到烈士陵园扫墓后又打着红旗顺路来参观汉画馆,熙熙攘攘而来,嘈嘈杂杂而去,扬起了满馆飞尘。孩子们见到了什么呢?我沉湎于回忆中:青年时代在法国留学,我的法语很差,听学院的美术史课只能听懂一半,很苦恼。有一回在鲁弗尔博物馆,遇到一位小学教师正在给孩子们讲希腊雕刻,她讲得慢,吐字清晰,不仅讲史,更着重谈艺术,分析造型,深入浅出,很有水平。我一直跟着听,完全听懂了,很佩服这位青年女教师的艺术修养。比之自己的童年教育,我多羡慕这些孩子们啊!最近几年,美育终于开始被重视。我希望,若干年后,那些难看的日用品和费了劲制造出来的丑工艺品将无人问津!

\subsection{扑朔迷离意境美}
我带小孙孙上街,他要买一把大刀,木制的,买了,一路耍弄,耀武扬威,得意非凡。快到家时,路边蜻蜓乱飞,他挥刀斩蜻蜓,累出一身汗,但一只也劈不着,几乎要哭了。我在路边树丛中仔细寻找,发现偶有停息在叶端的蜻蜓,轻手轻脚,不费劲捏住了蜻蜓的翅翼,小孙孙破涕为笑,连大刀也不要了。每到公园,他总要捡根枯枝作长枪,似乎总有他要追逐打击的对象,但水里的鱼儿不怕他打,他慢慢观察人家钓鱼,又闹着要买渔钩了。

中学时代,我爱好文学,莫泊桑的出人意料的情节、古典诗词的优美韵律、鲁迅杂文的凝练深刻……都曾使我陶醉,但未有机缘专学文学,倒投身绘画了。每当我用丹青追捕客观世界之美,有时得意,有时丧气,丧气时就仿佛小孙孙用大刀劈不着蜻蜓,怨刀无用。文学与绘画是知心朋友吧,谈情说爱时心心相印,但彼此性格不同,生活习惯大异,做不得柴米夫妻,同住一室是要吵架的。

立足于文学的构思,只借助绘画的技法手段来阐说其构思,这样的绘画作品往往是不成功的,被讥为文学的或文学性绘画,因未能充分发挥绘画自身的魅力。虽然人的美感很难作孤立的分析,但视觉美与听觉美毕竟有很大的独立性,绘画和音乐不隶属于文学。“孤松矮屋板桥西”“十亩桑荫接稻畦”“桃花流水鳜鱼肥”……许多佳句寓形象美于语言美,诗中有画,脍炙人口,但仔细分析,其中主要还是偏文学的意境美。如从绘画的角度来看,连片的桑园接稻田可能很单调;孤松、矮屋与板桥间的形象结构是否美还需具体环境具体分析:桃花流水的画面有时抒情,有时腻人,通俗与庸俗之间时乖千里,时决一绳,文学修养不等于审美眼力。

旅游文学涉及风俗、人情、史迹、地理、地质、宗教及科技等多方面的挖掘与开拓,文章的道路广阔,但其间写景确也是一个重要部分。触景抒情主要靠的是文学的吸引力,如用文字一味描写景色,往往不易成功,正犹如用木刀斩蜻蜓或枯枝击游鱼!各地旅游的讲解员除作导游介绍外还该讲什么呢?“美”确乎很难讲,于是拉扯些故事、传说或笑话。山崖上、溶洞中别致的抽象性形象美便总被戴上多余的面具:猪八戒背亲、吕洞宾理鞋、三姐妹……没听清故事或看不真猪八戒的游人还真着急,似乎白来一趟了。当我读游记文章,总感到这些风马牛的故事搭配没意思。我也不爱读那些不厌其烦地用了许多形容词的写景段落,强迫语词来表达视觉美感,弄巧成拙,甚至只暴露了作者审美观的平庸。丹青写景都忌描摹的繁琐,文学写景似乎更宜用概括手法发挥景中之美感因素,我国传统绘画中重写意、大写意,这是绘画的特色与精华,这个“意”的表达,倒更是文学之所长吧!

你要我画下客观景象的面貌吗,可以的,并可保证画得像,而且还并不太吃力。但“像”不一定“美”。要抓住对象的美,表现那倏忽即逝的美感,很困难,极费劲,我漫山遍野地跑,鹰似的翱翔窥视,呕尽心血地思索,就是为了捕获美感。那么多画家画过桂林,已有那么多精彩的桂林摄影,人们头脑里都早已熟悉桂林了,但作家、画家、摄影师仍将世世代代不断地表现桂林的美感,见仁见智,作者个人的敏感永远在更新,青罗带与碧玉簪不会是绝唱。犹如猎人,我经常入深山老林,走江湖,猎取美感。美感就像白骨精一般幻变无穷,我寻找各样捕获的方法和工具,她入湖变了游鱼,我撒网;她仿效白鹭冲霄,我射箭;她伪装成一堆顽石,我绕石观察又观察……往往我用尽了绘事的十八般武艺依然抓不到她的踪影。每遇这种情况,夜静深思,明悟不宜以丹青来诱捕,而力求剥其画皮,用语言扣其心弦,应针对的是文学美而不是绘画美。我每次外出写生,总是白天作画,夜间才偶或写文,有人说诗是文之余,我的文是画之余,是画之补,是画道穷时的美感变种。

绘画和文学都各有其意境美,但其境界并不相同。就说朦胧吧,印象派绘画画面的形象朦胧美与文学意境的朦胧含蓄不是同一性质的美感。识别文学与绘画扑朔迷离的意境美是写游记的基础,也同时体现作者的修养与偏爱吧!20世纪60年代初,一位海外知心的老同窗写信告诉我,说他看到了我的一套海南岛风光小画片,认为那只是风光,是旅行写生,是游记。他的批评给了我极深刻的印象。数十年来我天南地北到处写生,吃尽苦头,就不甘心只作游记式的作品。“游记式”成了我心目中的贬词,是客观记录的同义语吧,我追求表达内心的感受与意境,画与文都只是表达这种感受与意境的不同手段。我画得多,写得少。作为专业画家,不得不大量画,有情时奋力画,无情时努力练,不少作品缺乏深刻的感受。正因不是作家,没有写作任务,写不出的时候绝不硬写,但来约稿写文的居然也愈来愈多,真怕写言之无物的文章,自己就不喜欢导游的讲解,怎愿写令人生厌的乏味游记呢!
\subsection{美育的苏醒}
美术,美术,对美与术的相互关系之探索,耗尽了画家的年华。

儿童在生活中感受到形象之美,于是涂鸦,其术也幼稚,其美也鲜明。但进了美术院校,接受了如何准确描绘物象外貌的技术训练,奴役于形似、酷肖,顾不上美感了,对美感日益麻木。我看,从原始人的绘画形式到文艺复兴技法之日臻完备,至近代最终以抒发感情为核心的数千年绘画发展史,似乎就浓缩在从儿童到巨匠的艺术成长过程中。

但技途往往使人误入歧途。

学习如何表达对象之形貌,其法有限,而如何表达对象之美感,其道无穷,因关系着作者的审美观之发展。

“技”往往受制于工具之异,因工具之异而派生的不同之技,反过来影响艺之变异。油彩和墨彩在划分西洋画和中国画中曾起了标志作用,像一盘棋局中的楚河汉界。艺术不同于下棋,在工具和手法上宜废弃楚河汉界,须不择手段,择一切手段,唯求效果。

林风眠在东、西方绘画的跋涉中,认识到艺术本质之异同,于是他创办的国立杭州艺专只设绘画系,不分西洋画系和中国画系,兼容并蓄,学生必须双向学习,先通而后专。今日几乎所有的美术院校都分油画和国画系,国画系中更分人物、山水、花鸟等科,本意当是培养专家。在一个院校的四年学习中培养专家,正如想在一个狭小的庭院中种植苍松巨柏,连扎根的土壤也没有。

我终生从事美术教学,时时面临着政治、题材、感情真伪、形式法则、艺术规律、误人子弟等等问题的困扰,此中滋味,颇如鲁迅所说的腹背受敌,只能横站。我在《美术》1979年第5期发表《绘画的形式美》提出:“……而如何认识、理解对象的美感,分析并掌握构成其美感的形式因素,应是美术教学中的一个重要环节,美术院校学生的主食!”其时当然遭到各式各样的批判。二十年后的今天,见到中国美术学院“综合绘画工作室”毕业班教学成绩展览,令我十分鼓舞。教学中着力于“美”的启示,“美”的构成因素之剖析,从黑白、彩色、材质等不同角度去探索,殊途同归,使学习者直接地把握造型规律,明悟绘画之“技”只为了服役于“艺”这一本质问题。这样教学的核心着意于“眼”,有别于斤斤于“手”,眼是手之师。

来源于生活,是这一实验教学成果有别于西方艺术的自我面目,教师在教学中起了主导作用。“美”存在于自然中,又被自然的芜杂所掩隐,教师向学生揭示“美”,好比妈妈已将生硬的食物咀嚼过,以便孩子更易消化吸收。当孩子的胃更健壮了,他自己能消化生硬的食物了,我提出过“美术自助餐”的观点,即学院里应开设传统及西方的各种课程,由学生选修(学分制),学生各人各系,因材施教,因材自教,希望这将不止于设想。

今天见到中国美术学院首先实验了综合绘画教学,令我怀念我们最初在杭州艺专绘画系的学习历程。我以“美育的苏醒”为题写这篇短文,以铭感、纪念林风眠老师的教学思想体系,且深信,星星之火必将燎原。我似乎已见到美术教学的更璀璨的前景。
\subsection{绘画的形式美}
\subsubsection{美与漂亮}
我曾在山西见过一件不大的木雕佛像,半躺着,姿态生动,结构严谨,节奏感强,设色华丽而沉着,实在美极了!我无能考证这是哪一朝的作品,当然是件相当古老的文物,拿到眼前细看,满身都是虫蛀的小孔,肉麻可怕。我说这件作品美,但不漂亮。没有必要咬文嚼字来区别美与漂亮,但美与漂亮在造型艺术领域里确是两个完全不同的概念。漂亮一般是缘于渲染得细腻、柔和、光挺,或质地材料的贵重,如金银、珠宝、翡翠、象牙,等等;而美感之产生多半缘于形象结构或色彩组织的艺术效果。

你总不愿意穿极不合身的漂亮丝绸衣服吧,宁可穿粗布的大方合身的朴素服装,这说明美比漂亮的价值高。泥巴不漂亮,但塑成《收租院》或《农奴愤》是美的。不值钱的石头凿成了云冈、龙门的千古杰作。我见过一件石雕工艺品,是雕的大盆瓜果什物,大瓜小果、瓜叶瓜柄,材料本身是漂亮的,雕工也精细,但猛一看,像是开膛后见到的一堆肝肠心肺,丑极了!我当学生时,拿作品给老师看,如老师说:“哼!漂亮啊!”我立即感到难受,那是贬词啊!当然既美又漂亮的作品不少,那很好,不漂亮而美的作品也丝毫不损其伟大,只是漂亮而不美的庸俗作品倒往往依旧是“四人帮”流毒中的宠儿。

美术中的悲剧作品一般是美而不漂亮的,如珂勒惠支的版画,如梵高的《轮转中的囚徒们》……鲁迅说悲剧是将有价值的东西毁灭给人看,为什么美术创作就不能冲破悲剧这禁区呢!
\subsubsection{创作与习作}
新中国成立以来,我们将创作与习作分得很清楚、很机械,甚至很对立。我刚回国时,听到这种区分很反感,认为毫无道理,是不符合美术创作规律的,是错误的。艺术劳动是一个整体,创作或习作无非是两个概念,可作为一事之两面来理解。而我们的实际情况呢,凡是写生、描写或刻画具体对象的都被称为习作(正因为是习作,你可以无动于衷地抄摹对象)。只有描摹一个事件,一个什么情节、故事,这才算“创作”。造型艺术除了“表现什么”之外,“如何表现”的问题实在是千千万万艺术家们在苦心探索的重大课题,亦是美术史中的明确标杆。印象派在色彩上的推进作用是任何人否认不了的,你能说他们这些写生画只是习作吗?那些装腔作势的蹩脚故事情节画称它们为习作倒也已是善意的鼓励了!

当然我们盼望看到艺术性强的表现重大题材的杰作,但《阿Q正传》或贾宝玉的故事又何尝不是我们的国宝?在造型艺术的形象思维中,说得更具体一点是形式思维。形式美是美术创作中关键的一环,是我们为人民服务的独特手法。我有一回在绍兴田野写生,遇到一个小小的池塘,其间红萍绿藻,被一夜东风吹卷成极有韵律感的纹样,撒上厚薄不匀的油菜花,衬以深色的倒影,优美意境令我神往,久久不肯离去。但这种“无标题美术”我画了岂不被批个狗血喷头!归途中一路沉思,忽然想到一个窍门:设法在倒影远处一角画入劳动的人群红旗,点题“岸上东风吹遍”不就能对付批判了吗!翌晨,我急急忙忙背着画箱赶到那池塘边。天哪!一夜西风,摧毁了水面文章。还是那些红萍、绿藻、黄花……内容未改,但组织关系改变了,形式变了,失去了韵律感,失去了美感!我再也不想画了!

我并不认为外国的月亮比中国的圆,但介绍一点他们的创作方法作为参考总也应允许吧!那是20世纪50年代我在巴黎学习时,我们工作室接受巴黎音乐学院的四幅壁画:古典音乐、中世纪音乐、浪漫主义音乐和现代音乐。创作草图时,是先起草这四种音乐特色的形线抽象构图,比方以均衡和谐的布局来表现古典的典雅,以奔放动荡的线组来歌颂浪漫的热情……然后组织人物形象:舞蹈的姑娘、弄琴的乐师、诗人荷马……而这些人物形象的组合,其高、低、横、斜、曲、直的相互关系必须紧密适应形式在先的抽象形线构图,以保证突出各幅作品的节奏特点。
\subsubsection{个人感受与风格}
儿童作画主要凭感受与感觉。感觉中有一个极可贵的因素,就是错觉。大眼睛、黑辫子、苍松与小鸟,这些具特色的对象在儿童的心目中形象分外鲜明,他们所感受到与表现出来的往往超过了客观的尺度,因此也可说是“错觉”。但它却经常被某些拿着所谓客观真实棍棒的美术教师打击、扼杀。

我常喜欢画鳞次栉比密密麻麻的城市房屋或参差错落的稠密山村,美就美在鳞次栉比和参差错落。有时碰上时间富裕,呵!这次我要严格准确地画个精确,但结果反而不如凭感觉表现出来的效果更显得丰富而多变化,因为后者某些部位是强调了参差,重复了层次,如用摄影和透视法来比较检查,那是远远出格的了。

情与理不仅是相对的,往往是对立的。我属科班出身,初学素描时也曾用目测、量比、垂线检查等方法要求严格地描画对象。画家当然起码要具备描画物象的能力,但关键问题是能否敏锐地捕捉住对象的美。理,要求客观,纯客观;情,偏于自我感受,孕育着错觉。严格要求描写客观的训练并不就是通往艺术的道路,有时反而是歧途、迷途,甚至与艺术背道而驰!

我当学生时有一次画女裸体,那是个身躯硕大的中年妇女,坐着显得特别稳重,头较小。老师说他从这对象上感到的是巴黎圣母院。他指的是中世纪哥谛克建筑的造型感。这一句话,确启示了学生们的感觉与错觉。个人感受之差异,也是个人风格形成的因素之一。毕沙罗与塞尚有一回肩碰肩画同一对象,两个过路的法国农民停下来看了好久,临走给了一句评语:“一个在凿(指毕沙罗),另一个在切(指塞尚)。”而我们几十个学生的课堂作业就不许出现半点不同的手法,这已是长期的现象了吧!

风格之形成绝非出于做作,是长期实践中忠实于自己感受的自然结果。个人感受、个人爱好,往往形成作者最拿手的题材。人们喜爱周信芳追、跑、打、杀的强烈表情,也喜欢凄凄惨惨戚戚的程腔。潘天寿的钢筋水泥构成与林风眠的宇宙一体都出于数十年的修道。

风格是可贵的,但它往往使作者成为荣誉的囚犯,为风格所束缚而不敢创造新境。
\subsubsection{古代和现代,东方和西方}
原始时代人类的绘画,东方和西方是没有多大区别的。表现手法的差异主要缘于西方科学的兴起。解剖、透视、立体感等等技法的发现使绘画能充分表现对象的客观真实性,接近摄影。照相机发明之前,手工摄影实际上便是绘画的主要社会功能。我一向认为伦勃朗、委拉斯贵兹、哈尔斯等等西方古代大师们其实就是他们社会当时杰出的摄影师。这样说并非抹杀他们作品中除“像”以外的艺术价值。古代的伟大杰作除具备多种社会价值外,其中必有美之因素,也是最基本、最主要的因素。很“像”很“真实”,或很精致的古代作品不知有千千万万,如果不美,它们绝无美术价值。现代美术家明悟、理解、分析透了古代绘画作品中的美的因素及其条件,发展了这些因素和条件,扬弃了今天已不必要的被动地拘谨地对对象的描摹,从画“像”工作的桎梏中解放出来,尽情发挥和创造美的领域,这是绘画发展中的飞跃。如果说西方古代艺术的主体是客观真实,其中潜伏着一些美感,那么现代绘画则是在客观物象中扬弃不必要的物件叙说,集中精力捕捉潜伏其中的美,而将它奉为绘画的至尊者。毕加索从古希腊艺术中提炼出许多造型新意,他又将德拉克罗瓦的《阿尔及利亚妇女》一画翻新,改画成一组新作,好比将一篇古文译成各种文体的现代作品。这种例子在现代绘画史中并不少见,就如鲁迅的《故事新编》。

我国的绘画没有受到西方文艺复兴技法的洗礼,表现手法固有独到处,相对说又是较狭窄、贫乏的,但主流始终是表现对象的美感,这一条美感路线似乎倒被干扰得少些。现代西方画家重视、珍视我们的传统绘画,这是必然的。古代东方和现代西方并不遥远,已是近邻,他们之间不仅一见钟情,发生初恋,而且必然要结成姻亲,育出一代新人。东山魁夷就属这一代新人!展开周昉的《簪花仕女图》和波提切利的《春》,尤脱利罗的《巴黎雪景》和杨柳青年画的《瑞雪丰年》,马蒂斯和蔚县剪纸,宋徽宗的《祥龙石》与抽象派……他们之间有着许多共同感受,像哑巴夫妻,即使语言隔阂,却默默地深深地相爱着!

美,形式美,已是科学,是可分析、解剖的。对具有独特成就的作者或作品造型手法的分析,在西方美术学院中早已成为平常的讲授内容,但在我国的美术院校中尚属禁区,青年学生对这一主要专业知识的无知程度是惊人的!法国19世纪农村风景画的展出在美术界引起的不满足是值得重视的,为什么在卫星上天的今天还只能展出外国的蒸汽机呢!广大美术工作者希望开放欧洲现代绘画,要大谈特谈形式美的科学性,这是造型艺术的显微镜和解剖刀,要用它来总结我们的传统,丰富发展我们的传统。油画必须民族化,中国画必须现代化,似乎看了东山魁夷的探索之后我们对东方和西方结合的问题才开始有点清醒。
\subsubsection{意境与无题}
造型艺术成功地表现了动人心魄的重大题材或可歌可泣的史诗,如霍去病墓前的石雕《马踏匈奴》,罗丹的《加莱义民》,德拉克罗瓦的《希阿岛的屠杀》……中外美术史中不胜枚举。美术与政治、文学等直接地、紧密地配合,如宣传画、插图、连环画……成功的例子也比比皆是,它们起到了巨大的社会作用。同时我也希望看到更多独立的美术作品,它们有自己的造型美意境,而并不负有向你说教的额外任务。当我看到法国画家夏凡纳的一些壁画,被画中宁静的形象世界所吸引:其间有丛林、沉思的人们、羊群,轻舟正缓缓驶过小河……我完全记不得每幅作品的题目,当时也就根本不想去了解是什么题目,只令我陶醉在作者的形象意境中了。我将这些作品名为无题。我国诗词中也有不少作品标为无题的。无题并非无思想性,只是意味深远的诗境难用简单的一个题目来概括而已。绘画作品的无题当更易理解,因形象之美往往非语言所能代替,何必一定要用言语来干扰无言之美呢!
\subsubsection{初学者之路}
数十年来我作为一个美术教师确曾教过不少学生,但我担心的是又曾毒害过多少青年!美术教师主要是教美之术,讲授形式美的规律与法则。数十年来,在谈及形式便被批为形式主义的恶劣环境中谁又愿当普罗米修斯啊!教学内容无非是比着对象描画的“画术”,堂而皇之所谓“写实主义”者也!好心的教师认为到高年级可谈点形式,这好比吃饱饭后才可尝杯咖啡或冰淇淋!但我不知道从抄袭对象的“写实”到表达情绪的艺术美之间有没有吊桥!我认为形式美是美术教学的主要内容,描画对象的能力只是绘画手法之一,它始终是辅助捕捉对象美感的手段,居于从属地位。而如何认识、理解对象的美感,分析并掌握构成其美感的形式因素,应是美术教学的一个重要环节,是美术院校学生的主食!
\subsection{造型艺术离不开对人体美的研究}
文学表现人的内心,需要剖析心理,这是理所当然的,科学的。丑也好,善也好,挖掘人的心灵深处。美术表现人的形象,同样要解剖人,脱去外加的衣服,人本身是赤裸裸的!用形式表现外形,用文字刻画内心,都必须科学地研究人,本质的人。人生到世界上来,本来是赤裸裸的。人是很美的,当然人类并不认为自己是丑的,而觉得是美的。从造型角度分析,美的因素在哪里呢?人,是活的机体,有块有肢,有头、手、脚,有硬骨与软体,是极复杂又极微妙的完整结构。他能站起来,能坐下去,能躺倒,能斜着一只脚独立,能跑、跳、舞,能像燕子似的滑翔……动也好,静也好,其间都交织着形体结构的稳定感和运动感。这是造型美感的基础,它们是基于科学的生理结构的,否则就会站不稳、跃不起。我们观察千变万化的自然形态,比较各式各样的形象,有的看了感到舒服,有的不舒服,无论山峰、建筑、树木、家具……或厚重,或苍劲,或轻快,或苗条,这些感觉的产生,都是与人本身的生理机能结构分不开的。感觉的舒服基于生理机体的舒服。你坐得舒适了,外形就易给人舒适之感,因人们对坐得舒适与不舒适的感觉是共通的。同样,对一件物什,一把壶,一个罐子……高了、矮了、大了、小了……这也是人的自我感受。所以人们欣赏美,多半本源于人体本身的美。凡是违反了人的生理机能和动作规律,违反了人的基本结构,就感觉不舒服了,就不美。我每看一群树,犹如看一群人,观察它们之间的相互穿插、相互呼应和相互抱合的关系,这犹如邻居相争吵或朋友相叙旧的关系,当然我们听不到它们的声音,我们只从它们的形体上感到一切。这一切基于人体生理结构,基于人体美。造型艺术家要钻研人体美是天经地义的基本功。造型美的基本因素,如均衡、对比、稳定、变化统一,等等,都存在于人体中,在美术教学中,这方面的问题最明显,最易理解。

英国的雕刻家亨利·摩尔,他凿出来的人体已不完全是人的外表皮相,他表达了人的动和静、伸和缩,歌颂了宇宙主人的力量!有人说他的灵感来自东方,他从我们的假山石里获得了重大的启示。我却要反过来思考,那么我们的假山石又从哪里得来的启示呢?是从人体得来的。尽管设计假山石的艺人巧匠没有写生过人体,假山石的结构美是抽象的,但其起、伏、挑、擢的抽象美,并非是文艺之神阿波罗的恩赐,其实只是人们蹲、卧、前扑与回顾等生理活动的潜在的转化。即使这种潜在的关系被深深隐藏着,但在罗丹、摩尔这些有着长期丰富造型实践的艺术家看来是一目了然的,这是形式的科学。也许别人对假山某处多一块或少一块石头是无所谓的吧,但在大师眼中那却是生命攸关的脑袋问题!书法也一样,一撇一捺,骑稳没有?跨够了吗?或求严谨,或爱奔放,这些不同感情的体现依据的是人体机能,是人体美。吴道子作画前要看舞剑,搏斗启发了大草书,这些都不是唯心论吧!所以我认为造型艺术是离不开对人体美的研究的。

西方艺术发展的历史也不短了,遗产很丰富,除了中世纪以外,造型艺术的精华可以说大部分都存在于人体美之中。我可能讲得片面些。当然艺术的思想性和深刻的社会意义还是极重要的,但从造型美、形式美的角度来研究,自希腊、罗马以来,人体美是他们代代耕耘的美的沃土,开掘的美的矿源。不理解人体美,便无法体会西方美术。因此,在艺术教育中,除了继承发扬我国固有的造型体系外,同时要吸取外来的血液,人体这门课程是不可或缺的,而且要深入钻研,不只是点缀而已。即使有的学生日后不当人物画家,也要研究人体美。我自己画过半辈子人体,今天老了,只画风景,不画人体了,但说句良心话,我庆幸在长期人体研究中窥见了西方造型美的门径。

在艺术中表现裸体,同社会风俗要发生矛盾。这不仅在中国是如此,在西洋也是有这问题的,不过程度不同而已。就说希腊那样风行裸体艺术的时期吧,有一位经常当女裸体模特儿的叫芙罕内,就被控告有伤风化,被法庭拘捕审讯。后来开庭时,因她的出色美貌被免罪释放了。我国的封建意识根深蒂固,男女授受不亲。古代妇女生病,只能从帐子里伸出一只手来让医生按脉,甚至用一根线缚在脉门上由医生去摸线开方。我总记得鲁迅讲过,即使如何如何爱国,总不能掩饰我们落后的、反科学的东西。我们的人物画如不研究人体,必然越来越不行了,这与医生不解剖人体是一样的荒谬。像任伯年等许多人物画家是有才能、有功力的,但对人体的表现还只停留在概念的阶段,吃了很大的亏。我们的青年一代绝不满足只做任伯年的继承人吧!

我们的封建社会那么长,绝不让人们公开看裸体,因此见到裸体就不得了,这是现实问题,不得不考虑。这回北京油画研究会展览了几幅裸体,有的色彩斑斑,有的偏于变形,女裸的生理特征并不突出,但有一幅很写实,肤色体形就像躺着的女裸体摄影,展出期间围着这画看的人特别多,其中多半不是欣赏艺术,而是来看别处无法看到的女裸体的,这起了不好的副作用,有些观众提了十分生气和尖锐的意见。在湖南展出时,我也正好在湖南,曾有人为此向我提出质问。还听说,过去展出裸体时,有人拍了照当黄色图片去卖,所以这回油画研究会的展览只好禁止拍照。艺术学院教室里画裸体本是课堂作业,作业挂出观摩是成绩汇报,但学院里有许多青年临时工,他们见了可稀奇了,又造成了坏影响。这些情况都对我们研究裸体不利,是容易给我们抹黑的。为了珍惜我们对艺术严肃认真的探索,保卫我们刚获得的艺术创作自由,我希望纯习作性的裸体不必公开展出,尤其不要在公园等公共场所展出。破除封建的工作,还是要有步骤、有阶段地进行。鲁迅先生洗澡不避孩子这是实实在在的基本教育,突如其来地让今天的青少年看女裸体恐怕还是害多益少吧!
\subsection{关于抽象美}
对于美术中的抽象美问题,我想谈一点自己的理解。

有人认为首都机场壁画中的《科学的春天》是抽象的。其实,它只能说是象征的,它用具体形象象征一个概念,犹如用太阳象征权力,用橄榄枝象征和平一样,这些都不能称抽象。抽象,那是无形象的,虽有形、光、色、线等形式组合,却不表现某一具体的客观实物形象。

无论东方和西方,无论在什么社会制度中,总有许多艺术工作者忠诚地表现了自己的真情实感,这永远是推进人类文化发展的主流。印象派画家们发现了色彩的新天地,野兽派强调了艺术创作中的个性解放,立体派开拓了造型艺术中形式结构的宽广领域……这些探索大大发展了造型艺术的天地。数学本来只是由于生活的需要而诞生的吧,因为人们要分配产品,要记账,听说源于实用的数学早已进入纯理论的研究了;疾病本来是附着在人身上的,实验室里研究细菌和病毒,这是为了彻底解决病源问题。美术,本来是起源于模仿客观对象吧,但除描写得像不像的问题之外,更重要的还有个美不美的问题。“像”了不一定美,并且对象本身就存在美与不美的差距。都是老松,不一定都美,同是花朵,也妍媸有别。这是什么原因?如用形式法则来分析、化验,就可找到其间有美与丑的“细菌”或“病毒”在起作用。要在客观物象中分析构成其美的因素,将这些形、色、虚、实、节奏等等因素抽出来进行科学的分析和研究,这就是抽象美的探索。这是与数学、细菌学及其他各种科学的研究同样需要不可缺少的老老实实的科学态度的。

“红间绿,花簇簇”“万绿丛中一点红”,古人在绿叶红花或其他无数物象中发现了红与绿的色彩的抽象关系,寻找构成色彩美的规律。江南乡镇,人家密集,那白墙黑瓦参差错落的民居建筑往往比高楼大厦更吸引画家。为什么?我们曾斥责画家们不画新楼画旧房,简单地批评他们是资产阶级思想。其实这是有点冤枉的。我遇到过许多热爱祖国、热爱人民的老、中、青年画家,他们自己也都愿住清洁干燥、有卫生设备的新楼,但他们却都爱画江南民居,虽然那些民房大都破烂了,还是要画。这不是爱其破烂,而是被一种魅力吸引了!什么魅力呢?除了那浓郁的生活气息之外,其中白墙、黑瓦、黑门窗之间的各式各样的、疏密相间的黑白几何形,构成了具有迷人魅力的形式美。将这些黑白多变的形式所构成的美的条件抽象出来研究,找出其中的规律,这也正是早期立体派所曾探索过的道路。

谁在倒洗澡水时将婴儿一起倒掉呢?我无意介绍西方抽象派中各种各样的派系,隔绝了近三十年,我自己也不了解了。我们耻于学舌,但不耻研究。况且,是西方现代抽象派首先启示人们注意抽象美问题的吗?肯定不是的。最近我带学生到苏州写生,同学们观察到园林里的窗花样式至少有几百种,直线、折线、曲线及弧线等等的组合,雅致大方,变幻莫测。这属抽象美。假山石有的玲珑剔透,有的气势磅礴,既有平易近人之情,又有光怪陆离之状。这也属抽象美。文徵明手植的紫藤,苍劲虬曲,穿插缠绵,仿佛书法之大草与狂草,即使排除紫藤实体,只剩下线的形式,其美感依然存在。我在野外写生,白纸落在草地上,阳光将各种形状的杂草的影子投射到白纸上,往往组成令人神往的画面,那是草的幽灵,它脱离了躯壳,是抽象的美的形式。中国水墨画中的兰、竹,其实也属于这类似投影的半抽象的形式美范畴。书法,依凭的是线组织的结构美,它往往背离象形文字的远祖,成为作者抒写情怀的手段,可说是抽象美的大本营。云南大理石,画面巧夺天工(本是天工),被装饰在人民大会堂里,被嵌在桌面上,被镶在红木镜框里悬挂于高级客厅;桂林、宜兴等地岩洞里钟乳石的彩色照片被放大为宣传广告画,这都属抽象美。在建筑中,抽象美更被大量而普遍地运用。我国古典建筑从形体到装饰处处离不开抽象美,如说斗拱掩护了立体派,则藻井和彩画便成了抽象派的温床。爬山虎的种植原是为了保护墙壁吧,同时成了极美好的装饰。苏州留园有布满三面墙壁的巨大爬山虎,当早春尚未发叶时,看那茎枝纵横伸展,线纹沉浮如游龙,野趣感人,真是大自然难得的艺术创造,如能将其移入现代大建筑物的壁画中,当引来客进入神奇之境!大量的属抽象范畴的自然美或艺术美,不仅被知识分子欣赏,也同样为劳动人民喜爱。而且它们多半来自民间,很多是被民间艺人发现及加工创造的,最明显的是工艺品,如陶瓷的窑变、花布的蜡染等。人们还利用竹根雕成烟斗,采来麦秆编织抽象图案,拾来贝壳或羽毛点缀成图画;串街走巷的捏面艺人,将几种彩色的面揉在一起,几经扭捏,便获得了绚丽的抽象色彩美,他们在这基础上因势利导巧妙地赋予具象的人物和动物以生命。

抽象美是形式美的核心,人们对形式美和抽象美的喜爱是本能的。我小时候玩过一种万花筒,那千变万化的彩色结晶纯系抽象美。彩陶及钟鼎上杰出的纹样,更是人类童年创造抽象美才能的有力例证。若是收集一下全国各地区各民族妇女们发髻的样式,那将是一次出色的抽象美的大联展。

似与不似之间的关系其实就是具象与抽象之间的关系。我国传统绘画中的气韵生动是什么?同是表现山水或花鸟,有气韵生动与气韵不生动之别,因其间有具象和抽象的和谐或矛盾问题,美与丑的元素在作祟,这些元素是有可能抽象出来研究比较的。音乐属听觉,悦耳或呕哑啁哳是关键,人们并不懂得空山鸟语的内容,却能分析出其所以好听的节奏规律。美术属视觉,赏心悦目和不能卒视是关键,其形式规律的分析正同于音乐。将附着在物象本身的美抽出来,就是将构成其美的因素和条件抽出来,这些因素和条件脱离了物象,是抽象的了,虽然它们是来自物象的。我认为黄宾虹老先生晚年的作品进入半抽象的境界,相比之下,早期作品太拘泥于物象,过多受了物象的拖累,其中隐藏着的,或被物象掩盖着的美的因素没有被充分揭示出来,气韵不很生动,不及晚年作品入神。文人画作品优劣各异,不能一概而论,其中优秀者是把握了具象抽象的契合的。我认为八大山人是我国传统画家中进入抽象美领域最深远的探索者。凭黑白墨趣,凭线的动荡,透露了作者内心的不宁与哀思。他在具象中追求不定形,竭力表达“流逝”之感,他的石头往往头重脚轻,下部甚至是尖的,它是停留不住的,它在滚动,即将滚去!他笔下的瓜也放不稳,浅色椭圆的瓜上伏一只黑色椭圆的鸟,再凭瓜蒂与鸟眼的配合,构成了太极图案式的抽象美。一反常规和常理,他画松树到根部偏偏狭窄起来,大树无根基,欲腾空而去。一枝兰花,条条荷茎,都只在飘忽中略显身影,加之,作者多半用淡墨与简笔来抒写,更构成扑朔迷离的梦的境界。
\subsection{油画之美}
“远看西洋画,近看鬼打架。”这是我国人民最初接触西洋油画时的观感,这种油画大概是指近乎印象派一类的作品。我在初中念书时,当时刘海粟先生到无锡开油画展览,是新鲜事物,大家争着去看,说明书上还做了规定:要离开画面十一步半去欣赏。印象派及其以后的许多油画,还有更早的浪漫派大师德拉克罗瓦的作品,都宜乎远看。有人看德拉克罗瓦的《希阿岛的屠杀》觉得画中那妇人低垂的眼睛极具痛苦的表情,但走近去细看,只是粗粗的笔触,似乎没有画完,他问作者为什么不画完,德拉克洛瓦回答说:“你为什么要走近去看呢?”为了要表达整体的视觉形象,要表达空间、气氛及微妙的色彩感受,有时甚至有点近乎幻觉,油画大都需要有一定的距离去看,因之其手法与描图大异,只习惯于传统工笔画的欣赏者也许不易接受“鬼打架”式的油画。远看有意思,近看这乱糟糟的笔触中是否真只是鬼打架而已呢?不,其中大有名堂,行家看画,偏要近看,要揭人奥秘。书法张挂起来看气势,拿到手里讲笔墨、骨法用笔。中国绘画讲笔法,油画也一样。好的作品,“鬼打架”中打的是交响乐!印象派画家毕沙罗和塞尚肩并肩在野外写生,两个法国农民看了一会儿,离开时评论起来:“一个在凿,另一个在切。”所以油画这画种,并非只满足于远看的效果,近看也自有其独特的手法之美,粗笨的材料发挥了斑斓的粗犷之美,正如周信芳利用沙哑的嗓音创造了自己独特的跌宕之美。

印象派以前的油画,大都是接近逼真,细描细画,无论远看近看,都很严谨周密,绝无“鬼打架”之嫌。从我们“聊写胸中逸气”的文人画角度看,这些油画又太匠气了!油画表现力强,能胜任丰富的空间层次表达和细节的精确刻画,但绝不应因此便落得个“匠气”的结论。工笔与写意之间无争衡,美术作品的任务是表达美,有无美感才是辨别匠气与否的标准,美的境界的高低更是评价作品的决定性因素。就说摄影,照相机是无情的,一旦被有情人掌握,出色的摄影作品同样是难得的美术品。现在国外流行的超级写实主义,或曰照相写实主义,作品画得比照相还细致。“逼真”与“精微”都只是手段,作者要向观众传达什么呢?最近南斯拉夫现代绘画展览在北京举行,其中很多是超级写实主义的,比方画一棵老树的躯干,逼真到使你同意它是有生命和知觉的,它当过董永和七仙女缔订婚姻的媒人,它是爱情的见证,也是创伤的目击者。我曾在太行山区的一个小村子里见过一株巨大的、形象独特的古槐树,老乡说祖辈为造房子,曾想锯倒它,但一拉锯子,树流血了,于是立刻停下来。艺术家想表现的正是会流血的古树吧!公安人员要破案,必先保护现场,拍摄落在实物上的指印,凭此去寻找躲藏了的凶犯。照相写实主义的画家像提出科学的证据似的一丝不苟地刻画实物对象,启示读者去寻找作案犯,不,不是作案犯,去寻找的是被生活的长河掩埋了的回忆、情思和憧憬!这次全国青年美展中,有位四川作者用照相写实主义的手法画了一个放大的典型的四川农民头像,这位活生生的父老正端着半碗茶,贫穷、苦难和劳动的创伤被放大了,汗珠和茶叶是那样的分明……画面表现得很充分,效果是动人的,题目写的是《我的父亲》,在评选中我建议改为《父亲》,这个真实的谁的父亲正是我们苦难父辈的真正代表!

不择手段,其实是择一切手段,不同的美感须用不同的手段来表达,“鬼打架”的笔触也好,赛照相的技法也好,问题是手段之中是否真的表达了美感。我国传统绘画没有经过解剖、透视、色彩学等等科学的洗礼,表现逼真的能力不及油画强,但古代画家的创作大都是基于美的追求的。我们看油画,由于油画能表现得逼真,于是便一味从“像”的方面来要求、挑剔,似乎油画的任务只是表现“像”,不是为了美的欣赏。某些成见和偏见已习以为常了,认为中国画的山水、花鸟是合理合法的,而油画的专业只能画重大题材,如画风景花卉之类便不是正经业务,因之全国美展之类的大展览会是不易入选的。西洋的油画,早先确是主要用来表现宗教题材、重大事件和文学题材的,画家的任务主要是为这些故事内容做图解。时过境迁,大量的作品已过时,被淘汰了,能留下来的,除有些因文物价值之外,主要是依靠了作品本身的艺术价值——美。人们不远千里赶到意大利去看拉斐尔画的圣母圣子,绝非因为自己是基督徒。自从照相机发明以后,油画半失业了,塞翁失马,它专心一意于美的创造了,这就是近代油画与古代油画的分野,这就是近代西方油画倾向东方,到东方来寻找营养的基本原因。音乐是属听觉范畴的,好听不好听是关键。美术属视觉范畴,美不美的关键存在于形式之中。形式如何能美,有它自己的规律。画得“像”了不一定就美,美和“像”并不是一码事,表皮像不像,几乎人人能辨别,但美不美呢?不通过学习和熏陶,审美观是不会自己提高的。一般市民大都不能辨认篆刻之优劣,就是一个文学家,他也并不能因为是文学家便一定真正懂得美术。左拉与印象派画家们很接近,他与塞尚是同乡,从小同学,长期保持着亲密的友谊,但他很不理解塞尚,他将塞尚作为无才能的失败画家为模特儿写了一部小说,书一出版,塞尚从此与他绝交了。油画之美,体现在形和色的组织结构之中,如只从内容题材等等方面去分析,对美的欣赏还是隔靴搔痒的。

贵州人出差到上海,吃不惯清淡的偏甜味的菜,自己要带一罐浓烈的辣酱。同样,有人不喜欢雅淡的水墨画,爱油画的浓郁。梵高作画用色浓重,他在乡村写生,儿童们围拢来看,当他将大量浓艳的油彩从瓶里徐徐挤到调色板上时,儿童们惊叫起来,那游动着的彩色动物首先就具刺激性的美感,难怪有些画家的调色板本身就是令人寻味的。这种五颜六色的材料美已显示了油画之美,正如大理石和花岗岩自身的质感美已具备了雕刻之美。古代油画爱利用空间深度,总将画面表现得暗暗的,突出几处明亮的主要形象,但因为老是这种暗暗的老调子,今人贬之曰“酱油调子”,人们开始代之以鲜明的色彩,画面明亮起来了。总喝咖啡也单调,要换清淡的茶了,一向偏于浓、艳、厚重的油画也追求淡、雅、轻快的感觉了。古代油画宜配华丽的金框,颇像穿着大礼服的绅士,近代油画自由活泼多了,穿着衬衣便上大街了。油画这材料可画得特别浓重,也易于画得分外的亮堂,浓妆淡抹总相宜。

苍山似海,这说的是形式美感,苍山与海的相似之处在哪里呢?因那山与海之间存在着波涛起伏的、重重叠叠的或一色苍苍的构成同一类美感的相似条件。如将山脉、山峰及其地理位置的远近关系都画得正确无误时,山是不会似海的。画家们强调要表现那种波涛起伏的、重重叠叠的或一色苍苍的美,却并不关心一定要交代是山还是海,这表现的就是抽象美。西方油画中的抽象派席卷了画坛几十年,今天依旧余威不灭,它彻底解放了形式,不再受内容的制约。不过抽象还是从具象中抽出来的,即使其间已经过无数的转折与遥远的旅途,也还是存在着直接或间接的血缘关系。酒不是粮食或果子,但是是用粮食或果子酿成的。极端荒唐的梦,仍可分析出其生活中的逻辑根源吧!至于作品,同是抽象派,依然是优劣各异,感情的真伪有别,具体作品须具体分析,不能一概而论。

“一见倾心”的情况是确实存在的,我早年看陈老莲画的人物,就一见倾心,看梵高的画,也一见倾心,读者们都看过不少画,也遇到过使你一见倾心的作品吧!作为美术品,首先要争取观众的一见倾心。一见倾心决定于形式,但形式之中却蕴藏着情意。绘画,作为欣赏性的绘画,它的价值不是由其所表现的题材内容来决定,而由其形式本身的意境高低来决定。形式本身能表达意境?是的,形式本身能表达意境,这不同于所画故事内容方面的文学意境,这是造型艺术用自己独特的语言所表达的独特意境,这也正是能否真正品评一件美术作品的要害。同是裸体面,许多人看不出黄色与美感的区别,这不能不归罪于我们审美教育的落后。安格尔画裸体,无论是《土耳其浴室》的裸女群或《泉》这样的单个裸女,从那丰满的体态造型到红润的肤色,都表现了作者甜蜜蜜的世俗审美观。马蒂斯在女体中充分发挥了流畅与曲折的韵律美,莫底利安尼的女体刺激性特别强,他彻底揭示了人的野性和肉体的悲哀,他的画浸透着“邪气”的美,但还绝不是黄色的。虽都只是裸妇,都只是形式的探求,却透露了作者的审美趣味和思想感情,同时还显现了他们的时代背景。形式美的独立性发展到抽象性时,形式之中依然是有作者的灵魂在的,有人甚至说,在书法之中可看出作者寿命的长短呢!

“这画属什么派?”每同亲友们看西洋画,他们总会提出这个老问题。西方油画五花八门,看不懂时只好归到它们所属的派系里去存疑。古典派、浪漫派、写实派、印象派、野兽派、抽象派……画家们在对人生、对自然的探索过程中,或在题材内容方面,或在新的美感的发现方面,或在表现手法的创新方面,有共同倾向时形成过各式各样的派,但新的艺术感受很易彼此渗透,愈到现代,派的区别愈不明显,一人一派,个人特色却永远是作品的灵魂。看一幅风景画,那画的是黄山吧,我们要看的不是黄山,黄山的彩色照片多的是,我们要看的是从作者灵魂这面镜子里折射出来的黄山。毕加索永不肯做荣誉的囚犯,他经历过多种派,却很难说属哪一派,他一生都在做新的探索,为人类创造了巨大的精神财富。艺术传统的继承不同于张小泉、王麻子剪刀须依靠老招牌,在艺术上,儿子不必像老子,一代应有一代的想法,艺术上的重复是衰落的标志!管他什么派不派,只看作品的效果!

油画诞生于西方,所表现的思想感情和审美情趣都是西方的。西方的东西我们也吃,西方的油画我们也看。但是我们自己的画家画油画呢?他必有跟人学舌的阶段,但为了掌握语言后用以表达自己的感受,他的作品必定比西方名画更易为乡亲们喜爱,油画民族化是画家忠实于自己亲切感受的自然结果!我有一些从事美术的老同窗老相识,他们留在欧美没有回国,有些已是名画家了,他们在西方生活了数十年,作品中仍更多流露着东方意境。就说西方画家,吸取东方情调的也愈来愈多了,尤脱利罗的风景画更富有中国的诗情画意,他画那些小胡同的白粉墙,虽带几分淡淡的哀愁,却具“小楼一夜听春雨,深巷明朝卖杏花”的幽静之美。油画并不是洋人的专利,土生土长的中国油画没有理由自馁,祖国泥土的浓香将随自己的作品传遍世界,闻香下马的海外观众必将一天比一天多起来!有人不同意提倡油画民族化,认为艺术是各民族共有的,是世界性的。确乎,民族形式决定于民族的生活习惯,而生活习惯又决定于生产方式,生产方式一致起来,于是生活习惯也会接近起来,民族形式的差异亦将逐步泯灭。苗家姑娘种地时不便穿累赘的长裙子,只在节日或拍电影时再专作盛装打扮。油画的民族化,实际上也正是逐步在消灭东西方的隔膜和差异,促进东西方人们感情的融洽。历史在演变,我们生活在历史演变的一定时期中,只能为这一时期服务吧!
\subsection{摄影与形式美}
“你这画简直是照相。”我们艺术院校的学生常这样讽嘲照相式的作业。回忆自己的学生时代,也最瞧不起照相似的画,如果谁说我的画像照相,认为那是莫大的侮辱。

初学画不久,感兴趣的是色彩的跳跃,笔触的奔放,这些才叫艺术啊!因为不喜欢照相似的光滑细腻的描写技法,就连照相也不喜欢了。知识慢慢积累,兴趣逐渐扩大,体会到造型艺术手法的多样是缘于表现不同的情调和意境。我讨厌的其实是一味抄录对象,没有情调,缺乏意境的作品,而不该迁怒于表现手法的本身。绘画是人的感情的表现;如果摄影机被人利用,摄影也同样是作者感情的表现了。

我这个原先不喜欢摄影的人曾经特别关心起摄影来,那是“四人帮”控制期间,美术被迫愚蠢地描摹自然,硬要跟摄影比赛。我看到国内外摄影技术飞速发展,作品不仅生动活泼,而且愈来愈美,它闯入美术的园地里来了!我竭力赞扬摄影,是怀着一种私念的:摄影作品的质量已远远超过“写实”的绘画了,绘画往哪里走!逼上梁山,绘画该讲形式美了,被压在雷峰塔下的形式美能否早一天获得解放,我祈祷雷峰塔的倒掉!

“四人帮”倒后,情况确是好起来了,形式美开始受到注意,不仅在绘画中如此,在许多摄影作品中也愈来愈多地在发挥形式美的威力了。我说“威力”,并不过分,一切造型艺术都依赖形式赋予躯体,形式是否美,关系到人们爱看不爱看的大问题,这决定作品的命运。摄影主要是为具体的社会任务服务的吧,重大事件的记录、肖像的留念、证件的依据……但同时摄影也已成为以欣赏为主的美术作品。我曾经比方美术和文学有血缘关系,而美术和摄影只是同院的邻居,但现在看来这两家邻居将结成新的亲家了。不是吗?超级现实主义的绘画大量吸取了摄影的手法,摄影又在吸取油画及水墨大写意的手法,你吸取我的,我吸取你的,因为目标共同起来了,这个目标就是表达美的意境,因此也就有了相同的甘苦——对形式美的探求!

妈妈领着两个孩子到公园,一个孩子有所发现了,兴奋地叫起来:“这里真好玩,刺丛里也开花。”他指的是玫瑰。另一个孩子也有所发现了,却告诉妈妈:“这里不好玩,花丛里都是刺。”他指的也是玫瑰。我们欣赏玫瑰,拍摄过玫瑰,画过玫瑰,感到玫瑰是美的,难怪姑娘们喜欢将自己的脸庞依靠着玫瑰摄影!然而孩子们的观察却提醒了我对形式美的进一步分析。玫瑰花,质感柔软的圆圆形,圆圆的花朵被托在那放射着尖尖针刺的坚硬枝条间,二者组成了强烈的对比美。如果画面以花为主,衬以带刺的枝,这是一幅以圆为主、线为配的抽象图案;如果画面以多刺的枝为主,尽量突出其丛丛针刺,间以花朵,这是一幅以乱线为主,配以圆圈的抽象图案。这两幅画面的形式结构是完全不同的,虽都是玫瑰,却体现了作者不同的思想情感,表达了不同的意境。迎春花开,那长长的缠绵的枝条间渐渐吐露出星星点点的多角形小黄花,“乱”的长线与“乱”的散点交错组成了变化多端的情趣:盈盈含笑啊、眉飞色舞啊,如夏夜的星空,似东风梳弄的垂柳……如何捕捉和表达这些不同的感受呢?关键就在点、线组织的疏密之间,点、线、面……这些形式的构成因素,也正同时是传递情感的青鸟吧!秋来叶落,衬着蓝天,光秃的树枝分外醒目,那线组织的交叉变化确是学画者的画谱。我经常围着一株野树团团转,双目紧紧盯住那枝杈,移步换形,欲追踪其不同的表情。“删繁就简三秋树”,谁删的?郑板桥删的。

法国现代雕刻家马约尔做雕塑时,往往用蜡烛光在雕塑的人体上到处寻找不必要的坑坑洼洼,将它填满,他强调形体的饱满。马约尔的作品特色是壮实和丰满,他的人物造型尽量向外扩张,使之达到最大的极限,再过一度便属臃肿或者就崩裂了,他追求的是量感美。人们都欣赏质感美,摄影师和画家经常喜欢表现彩陶与玻璃、粗布与丝绸等等粗犷与细腻间质感的对比美。但量感美,似乎易被忽视。量感美包含着面积、体积、容量和重量感等因素,是由长短比例及面积分割等形式条件构成的,它对形式所起的作用远比质感美显著。唐俑胖妞妞,隋俑坚而瘦,它们的量感美比之木雕或泥塑的质感美更突出。杭州灵隐寺前飞来峰有个大肚弥勒佛,笑得乐呵呵,游人都爱抱着他留影,结果遮住了佛的体形,破坏了量感美。我不知用什么方法可摄出其量感美来,至少要尽量压缩排除佛身以外的所有空间,让佛独占画面的全部面积!画家表现对象的量感美时,要夸张就夸张,要扬弃就扬弃。我不懂摄影,摄影师自然也有自己独特的手法,马约尔利用烛光,摄影中的光比马约尔的烛光要复杂多了吧!

《拉郎配》这个戏大家看过,很有意思。摄影里经常拉花配,我很不喜欢。拉来的花总是配在画面的最前景,很突出,虚情假意的手法令人一眼就看穿。其实作者是有苦心的,因为画面前景太空,不得已而为之,至于认为加了花朵就更美些,那是属于他自己的审美趣味了!“前景太空”,这是构图中一个要害问题。无论是绘画或摄影,画面中面积愈大的部分,它所起的形象效果也就愈大。“画龙点睛”适用到现代造型艺术中来时,关键是画龙,那庞大的龙的身形是主体,是决定形式美的要害。如那大面积的龙身处理得不好,任凭你点上珍珠玛瑙的眼睛也是徒然,古希腊雕刻家凿出完美的人体后,不点眼珠。北海公园的白塔总吸引人,那是风景中的眼睛吧,但包括白塔的一幅风景摄影能否成功,关键不在白塔,而在占画面绝大部分面积的前景,那是龙身,这龙是虚是实,必须具有自己的体形和气概,作者岂能以廉价的“拉花配”来聊以替代!一般情况下,远景好看,因为到远处景物重叠了,形象互相补充,容易显得丰富多样。令人棘手的是近景,由于透视作用,近景在画面上总霸占着大面积,它的形象又往往是空洞乏味的。天哪!我在风景画中奋斗数十年来经常碰面的死对头就是这个近景。传统山水画的构图讲究“起、承、转、合”,一般认为最困难的是“起”,即近景也。因此,选景首先是近景问题,要摄影白塔,在北海里围着它转,关键也是近景问题。我总是以百分之八十的精力来对付近景的,我喜欢画倒影,除了对诗意等等的追求外,在形式上讲,倒影是易于解决、丰富近景的大片面积的。小街小巷往往入画,因为近景狭窄,门窗相挤,形象变化多样,比近景空旷的大街大马路易于处理。“拉花配”本来是为了丰富近景,但许多“漂亮”的局部拼凑不起美的整体来!法国现代风景画家尤脱利罗专画巴黎街道。近景往往只是一些人行道、旧的石子路、水泥墙、台阶……他着意刻画近景的质感,那斑驳的残痕透露出年华易逝的淡淡的愁绪。这使我想起我国山水画中“远山取其势,近山取其质”的经验之谈。表现质感,那是摄影之所长,我见过不少表现沙漠的摄影,很精彩,那浩瀚的气概缘于单纯统一与微妙变化的结合,这往往使画家感到束手无策,而摄影于此出色地解决了近景空洞的问题。不过单靠表现沙漠的质感不一定就美,动人的作品还由于把握了沙漠波涛神秘的线组织美,就是具有了形式美的身段。

我年年背着画箱走江湖,江湖上常常遇到一些摄影工作者,多半是业余的,他们都说要学点画,主要是要学取景和构图。绘画常表现作者心底的形象,摄影总是拍摄具体的客观形象。当然摄影剪接等等手法日益发展,摄影作者将争取到更大的自由。不过对初学者来讲,如何在客观对象中表现主观的感情,第一个碰到的恐怕也就是取景和构图问题。构图问题是画面安排的形式问题,要在造型艺术的领域里工作,首先要重视形式,探讨形式美的独立性和科学性,不要怕“形式主义”的棍子。也许是职业病吧,我是经常地、随时随地以探寻形式美的目光来观察自然的,无论是一群杂树、一堆礁石,或是漩涡,或是投影……只要其中有美感,我总是千方百计要挖掘来为自己所用,它们甚至往往成为我画面构图中的主角。我发觉形式之中有意境,从石头的伏、卧中透露着作者的情思,像阿诗玛这样的石头其实已是太直率的比喻了!我无法举出多少条构图的规律,因取景和构图的目的是突出作者的意境,岂能以有限的规律来约束无限的意境。我曾对自己的学生说过:主要形象应占领画面的主要位置,即画面的中央。但这个“主要形象”不易理解,有时“云”“雾”等虚的部分正是主要形象,因此画面仅有的一些松、石之类的实体形象反而应靠边站了。
\subsection{风景哪边好?——油画风景杂谈}
寻寻觅觅,为了探求美,像采蜜的蜂,画家们总奔走在偏僻的农村、山野、江湖与丛林间。他们不辞辛劳,只要听说哪里风景好、人物形象好,交通最艰难最危险的地区也总有画家的脚印。西双版纳去的人太多了,四川的九寨沟又引起了莫大的兴趣。今年夏天,我背着笨重的油画工具到了新疆,在乌鲁木齐遇到了不少从北京、杭州等地也背着笨重的油画箱来的同道,一样的风尘仆仆,一样晒黑了的手脸,大家一见如故,心心相印,亲密的感情建立在共同的甘苦上。又三天汽车,我越过辽阔的戈壁,来到北疆的边境阿勒泰,在遥远的阿勒泰又遇到了背着油画箱的李骏同志,他乡遇故知!

童年爱吃的食物永远爱吃,我对青年时代受其影响较深的画家也总是眷恋难忘。印象派的莫奈、毕沙罗、西斯莱等人的作品曾经使我非常陶醉,但后来又不那么陶醉了,觉得他们对构图的推敲和造型的提炼不够重视,但色调清新和用笔的轻快还是使我喜爱的。“喜新厌旧”,我爱上塞尚了,他的作品坚实、组织严谨,冷暖色的复杂交错有如某些带色彩的矿石。他的画多半不是一次即兴完成的(某些晚年作品例外),由于反复推敲,用色厚重,设色层次多,画面显得吸油而无光泽,寓华于朴,十分沉着,具有色彩的金石味。但无论对静物、对风景、对人物,塞尚一视同仁,都只是他的造型对象。作者排除了一切文学意图的干扰,这点对中国的学画青年来讲(包括我自己)开始并不理解,不如吞饮印象派甜甜的奶那样适意可口。塞尚的美是冰冻的,与之相反,梵高的热情永远在燃烧,他的情融于景中,他的景是情的化身,他的作品大都是在每次激情冲动中一气呵成的,画面光辉通亮,迄今光泽犹新,像画成不久。提到油画风景,我立即就会想到尤脱利罗,他那笼罩着淡淡哀愁的巴黎市街风景曾使我着迷,他的画富于东方诗意,有些以白墙为主的幽静小巷,很易使我联想到陆游的诗:“小楼一夜听春雨,深巷明朝卖杏花。”尤脱利罗不仅画境有东方诗意,其表现手法也具东方特色,他不采用明暗形成的立体感和大气中远近的虚实感,他依靠大小块面的组合和线的疏密来构成画面的空间和深远感。以上这几位画家,都是我青年时代珍视的老师,我是经常在他们门下转轮来的,但他们之间又是多么的不同啊!

“道可道,非常道”,传道不容易,传绘画之道尤其困难。作品被人们接受了,作者创作的道路被承认了,于是大家跟着走,规规矩矩地跟着走往往易成为盲目地跟着走。你向哪里走啊?因为艺术的目标不是模仿,是创造,一家有一家的路——思路,鲁迅说路是鞋底造成的。今年在新疆,我早上四点多起床,画了一幅乱石溪水滩上的日出,好些同道说色调很美,但我感到落入了印象派的老题材和旧手法,自己六十多岁了,面对自家江山,还学人口舌,感到很不舒服。路跑得远,地方看得多,确能增长见闻,多得启发,但并不等于就开辟了艺术的新路。我以往每到一地写生,感到很新鲜,一画一大批,但过后细看,物境新鲜(相对而言),画境并不新鲜。初到青岛,一味画那些碧海红楼,一到苏州就离不开园林,风景画似乎离不开名胜古迹的写照,或者只是图画的游记。在河北农村住了几年,由于天天在泥土里干活,倒重温了童年的乡土感受,留意到土里的小草如何偷偷地生长,野菊又悄悄地开花,树,哪怕是干瘦的一棵树,它的根伸展得多么远啊!风光景色渐渐不如土生土长的庄稼植物或杂树野草更能吸引我了。我于是怀念起塞尚后期扎根在故乡作画的故事,我曾专程去访问他的故乡埃克斯,围着他反复表现的圣维多利亚山观察,那也不过是寻常的法国南方景色,但孕育了一个伟大的塞尚。我也曾在黄山观察,发现许多曾经启发过石涛的峰峦树石,没有石涛,这些峰峦树石对我并不如此多情。尤脱利罗生长在巴黎,终生都画巴黎的市街,作品浸透了对巴黎的爱,有人问他如果告别巴黎他将带走什么,他说只带一点巴黎古老墙壁上的灰末。还是王国维说的,所有写景其实都是写的情,他说的是文学写景。绘画写景也同样是写情,以形写情,其中有特殊的复杂性,有自己的科学规律。

有两棵松树,实际高度一样,都是五米,但其中一棵显得比另一棵高得多。显得高的那棵主干直线上升,到四米处曲折后再继续上升一米;显得矮的那棵主干在一米处便屈胁了,然后再上升四米,像一个腿短而上身比例太长的人,予人的感觉是基础太矮,上升得很吃力。而主干高的那棵显得上升并不费劲,绰有余劲,所以显得高。如果将这两棵松树的主干用条抽象的曲折线表现出来,曲折在高处者那线具上升感,而曲折在低处者那线具下垂感。在授课中,有学生因画不出松树的高度而怨纸太小了,我偏要他在火柴盒上画出这棵高高的松树来,关键是要分析对象的形式特点,突出其形式中的抽象的特点。“美”的因素和特色总潜藏在具象之中,要拨开具象中掩盖了“美”的芜杂部分,使观众惊喜美之显露:“这地方看起来不怎么样,画出来倒很美!”小小的山村色块斑斑,线条活跃跳动,予人生气勃勃的美感,如果捕捉不住其间大小黑白块面的组织美及色彩的聚散美,而拘谨于房屋细部的写真,许多破烂的局部便替代了整体的美感。这一整体美感的构成因素属抽象美,或者颠倒过来叫“象抽”,也可以说是形式的概括。总之是必须抽出构成其美感形式的元素来,这种元素的的确确的存在正是画家们探索的重大课题。我们大都看过老国画家作画,尤其是作泼墨写意时,一开始,黑墨落在白纸上,或成团块,或墨线交错,或许画的是荷花?石头?老鹰?不意却是屈原!有时,刚落墨数笔,还根本没有表现出是什么名堂,老画师就说不行了,他立即撕毁了画纸。未成曲调先有情,未备具象先有形。是鹰是燕,固然要交代清楚,但鹰与燕的身段体形或其运动感更是作品美不美的决定性关键,在未点出鹰与燕的具象时老画师在墨的抽象形式中已胸有成竹地把握了美与丑的规律。油画风景,山山水水、树木房屋……这些具体形象的表达并不太困难,而这些具体物体间抽象形式的组织结构关系,即形的起、伏、方、圆、曲、直及色的冷暖、呼应、浓缩与扩散,等等,才是决定作品美丑或意境存亡的要害。我有过一段难忘的回忆,那是在农村劳动期间,长期住在农民的家里,老乡们总是十分淳朴的,把我当成一家人看待,问寒问暖,很是亲切。每当我在田野画了画拿回屋里,首先是房东大娘大嫂们看,如果她们看了我的画感到莫名其妙,自己是一种什么滋味啊!我竭力要使她们懂!当她们说这画里的高粱很像时,她们是赞扬的,但我心里并不舒服,因为这画固然画得像,但画得并不好,如鱼饮水,冷暖自知,我不能欺蒙这些老实人!有几回,当我画得比较满意时,将画拿给老乡们看,他们的反应也显得强烈起来:“这多美啊!”在这最简单的“像”和“美”的赞词中,我了解了老乡们具有的朴素的审美力,即使是文盲倒不一定是“美盲”。当然,并不能以他们的审美观作为唯一的标准,我对自己的作品私下提出过这样一个要求:“群众点头,专家鼓掌。”我眷恋自己的土地和人民,我珍视艺术的规律,对抽象美是应做科学的分析和研究的,我努力探索寓抽象于具象的道路,真理有待大家不同的实践来检验!

去年在普陀山,遇到青岛啤酒厂的同志们远道去采购大麦。酿酒要选好粮食、好水源,创作文艺作品要生活源泉。“四人帮”倒台后,我与国外的同道旧友们恢复了通信,我为自己被封闭见闻数十年感到有些懊恼,但我能长期在辽阔的祖国大地上匍匐和奔驰又感到很充实,我劝老友们回国来探亲,探亲事小,回来再感受一下伟大母土的芬芳将使他们的艺术起质的进展!风景哪边好?祖国好,故乡好,感情深处好!我倒并不是狭隘的乡土观念者,事实上我几乎每年都有几个月是生活在边远地区或深山老林里,也几乎将踏遍名山大川了。不过数十年来写生经验的总结愈来愈感到已不是华丽的名胜在吸引我。踏破铁鞋,我追寻的只是朴实单纯的平常景物,是极不引人注意的景物,但其间蕴藏着永恒的生命,于无声处听惊雷!嘉陵江三百里山水呼唤我扑向前去,而滨江竹林里的雨后春笋更令我神往。别人给我介绍桂林的七星岩、叠彩山……我中意的倒是那边如镜的梯田。常有同道们要外出写生时征求我的意见,问哪里好,说海南岛吧,海南岛什么地方好?什么景好?从什么角度画好?黄山呢?去的人更多了,迎客松应如何表现?先有概念,再去对号,带某种“成见”去辨认大自然,这种作画的方式,我给取个名吧,叫“按图索骥”。我没有理由反对别人按图索骥,他们也许终于真的索到了骥。我自己的体会还是习惯于伯乐相马,大自然丰富、繁杂,永远开发不完,情况总是千里马常有而伯乐不常有,画家的“慧眼”远比其“巧手”更珍贵!

巧手还是要的,技法的多种多样是缘于作者思想感情的差异,思想感情在不断变化,技法也就在无穷无尽地增生。“洋画片”(指二十世纪三四十年代市面上流行的一些西洋画片),这名已含有贬义,贬它什么呢?我看并不因为它画法的细腻或审美的通俗。技法的各异是无可非议的,主要是缺乏意境,只停留在“景”的低级阶段。景中有情当然就不能局限在一个死角度徒然将景物来仔细描摹。我曾带领学生下乡写生,雨天不好活动,同学们采来大束野花在室内写生,这引起我要在野地直接写生野花的欲望,我不用手去采集野花,我用眼睛采集、组织那些长伴杂草和石头的精灵们!特别是野菊之类的小花,那是鬼闪眼的星空,似乎还发出大珠小珠落玉盘的铿锵之声!很难说我这画面算花卉、静物或风景。从情出发,题材内容可不受局限,静物、风景、人物之间的界限也均可打破,郭味蕖先生就曾探索花卉和山水的结合,体裁永远在演变!

“你这用的是纯黑?”我在外地写生中经常遇到当地美术工作者向我提出这个带着惊讶的疑问。不少人认为油画不能用纯黑和纯白,为什么不能,不知道,好像是老师说过不准用吧!印象派一味追求外光效果时是排除了纯黑与纯白的,在特定条件下追求特定效果时他们是有理由的。我也吃过印象派的奶,尊重这位百年前的老奶妈,但老奶妈讲的话有些已过时,只能当掌故听了。两千年来中国画家侧重黑与白的运用,创造了无数杰作,西方近代画家佛拉芒克大量用纯黑纯白,马尔盖画的码头也偏重于黑白的笔墨情趣,油彩的黑与中国的墨终于结成了亲家。其他技法方面的许多问题,如写意与写实,提炼或概括,从淋漓尽致的刻画到谨毛而失貌……中西画理的精华部分其实都是一致的,艺术规律是世界语,我希望大家不抱成见,做些切切实实的研究。
\subsection{美术自助餐}
聪明敏感的学生偏偏碰上迟钝固执的教师,这情况在美术老师教学中经常遇到。有些老师年年给吃陈旧甚至发霉的食物,年轻学生胃口健,食欲旺,总嫌吃不饱,不满足。记得我们学生时代,正当抗日战争期间,人事常变,教师流动频繁,接触到各式各样的教师,其中很多不称职,我们多次给校方提意见,就为了罢那些蹩脚教师的课。如遇到高水平的教师,便十分崇敬,爱护之情胜于对父母。艺术教学是感情教学,除技术外,更关键的是启发,所谓因材施教。教师的导向往往影响学生一辈子的道路,曾经误人子弟的教师,不知有多少!

古今之争、中西之争、流派之争,统统容纳于百家争鸣中。欲争,先鸣,鸣者,实践也,拿出实践的成果来,任人评比,无须争吵。中国画系强调从自己传统中发展,油画系强调要吸收外来的优秀文化,都属堂堂正正的宣言,无可批驳。至于“保守”“狭隘”“崇洋”虽已从潘多拉的匣子里飞了出来,只被视为暗流,但这暗流啊,却在强劲地奔流。

潘天寿强调中西要拉开距离,林风眠主张中西融合,他们各自都做出了独特的贡献,共创了中国现代美术的辉煌。但他们的教导曾在我这个年轻学生的脑海里遭遇、较量。正因他们都是我崇敬的导师,他们才能在我的脑海中搏斗,促使年青的一代独立思考、探索、追求。年轻人能在大师们的终点起跑,真是莫大的幸运。而那些庸庸碌碌的教师们,乱指方向或根本无方向可指的教师们则终于被遗忘了。

如今美术院校都设中国画、油画、版画等系,其实是事先规定了每个年轻人的前途,有点拉郎配的性质,忘记了艺术是发展感情的事业,需恋爱自由,但必须到一定阶段才能进行自由恋爱。因此院校中宜设多种素描、油画、国画、雕塑等工作室,并同时应选修雕塑、建筑、音乐、文学等等课程,这样每个学生自己属于自己的系,就像在丰盛的自助餐前,各自依照自己的胃口搭配自己的营养。那些谁都不吃的菜则必然渐渐被淘汰,促其淘汰。四五年的大学学习只能也必须打下广泛的审美基础,到研究班时再对专业深入研究。四十年、五十年、六十年风雨中能成长一位杰出的艺术家便是国家的荣幸,美术院校只是苗圃,绝不可能是艺术家的速成班,不要培养侏儒!
\subsection{无心插柳柳成荫——中国画创新杂谈}
常有昔日的学生及一些年轻人来信请教中国画的创新问题,我自己也创不好,怎能答复,谁又能开出创新的方案呢?都在努力创新,在探寻各式各样的新手法,想出奇制胜者尤多。新手法新样式固然也促进艺术内涵的递变,但技的演变若非缘于情之生发,一味为标新立异,则有意种花花不开,技中求艺,是缘木求鱼。无心插柳柳成荫,倒是符合艺术诞生的规律,柳插入了宜于生根的水土中,人们珍视水土!

虽是纸上谈兵,仍需探讨我们古老的绘画传统如何抽发出新枝来。大家早已认识近亲繁殖之恶果,如何吸取外来营养是传统健康发展的关键问题。伟大传统历史悠久,内容博大,但只求继承,还是比较单一的,有案可查,有例可循,要做到继之承之而不走样并非不可能。不守家规,爱上远方来客,同外国联姻生个漂亮的混血儿是喜事,但在艺术中杂交而生出出色的混血儿来却困难得多,然而新生的混血儿一代将是世界艺坛上强劲、活跃、健康的一代,明天是属于他们的!有东方父亲和西方母亲的混血儿,也有东方母亲和西方父亲的混血儿。东、西方艺术的融会与结合再复杂多样,不限于油彩与水墨之差异,不限于写实与写意、体面与线条、绘制与书写……千里之行始于脚下,从脚下谈起。从总的方面看,中国画大都着重用线造型,完成轮廓是绘事之本。印象派否认线之存在,认为物与物相碰或相托都凭色相及明度的差异,其间并没有线,线只是人为的界线。他们所见的全是空间世界中物与物的关系,不着眼孤立的物象,一味强调空间气氛中色相之美感。由此观之,中国画在纸的平面上用线画出清晰的形象,白纸上出现一个形象,形象是相对孤立的,与白纸背景并无严格的制约。无环境制约,突出了剪影式的形象,往往很醒目,但手法毕竟太单一,面对千变万化的客观世界,表现的能量极有限。古代的范宽、近代的龚贤体会到环境深远与体面厚实的重要,他们利用惯用的线之结构与笔触来制造厚实与深远感;米芾用墨点之浓淡来渲染空间层次;虚谷在线的继续中求其苍茫,竭力使形象融入无尽的空间里,他利用白背景做统一基调,使形象与背景浑然一体,避免了剪影式的单薄感。这些有创造性的杰出作者们在工具的局限性中竭力展拓表现手法,丰富画面,引深意境。他们在时代的局限中开辟了田园,艰辛地收获了自己耕作的粮食,我们吃其老本?参照印象派及其他西方许多有贡献的流派,参照雕塑、音乐、建筑、摄影……我们更能体会古代大师们用心之良苦,敬佩他们,但我们已处于更有利的条件之中,子孙已从爷爷的孤陋中走出去!

杰出的作品不受时代的淘汰,印象派否定不了我们的线之特色,比印象派更年轻更新的西方画派吸取了东方的线与韵。谁也说不清混血儿最早诞生于东方还是西方,怕只怕混血儿偏偏吸取了父母的缺点。自意大利文艺复兴以后,西方绘画充分发展了写实的本领,杰作无数,但杰作之所以杰出,主要是由于出色地表达了美感意境,若只论逼真,则逼真的作品太多太多了,未必动人。作为一个画家,必须充分掌握表现物象的基本能力,十八般武器件件拿得起来。但演出中常靠短打,在舞台上如何组织搭配得恰到好处,这是艺术。我是从实践中多次体会到这种甘苦的。早年学画水彩,有一回家里买来两条极新鲜的鳜鱼,湿漉漉,水淋淋,黄与黑的斑纹是那样的夺目。鱼等着下锅,我抢着画,抢那点水汪汪的色之美感。家里干脆说鱼暂时不吃,让我慢慢画。我于是另换一张较大的纸,仔细刻画起来,最后画成了鳃、鳍历历可数,眼目鼓鼓的两条死鱼。当然,并不是一概不能细画和具体刻画,有时美感正隐藏于繁杂之中,不深入刻画它是不肯显现的。

前年,非洲塞内加尔的挂毯到北京展出,引起了美术界的喝彩。毕加索真厉害,他有一双洞察各类造型美领域的慧眼,他发现了非洲民间艺术的强劲风格,为之拜倒,歌唱,追踪。通过他的再创造,人们更看清了非洲艺术的特色,他起过反射非洲艺术美的镜子的作用。塞内加尔挂毯展是现代作品,是他们传统的继承与发展,但其间看得出也有毕加索的伴奏。今年在北京展出了陕西渭北地区的拴马石,粗犷、率直、纯真,又引起了美术界的喝彩。然而许多出色的民间艺术已不为民间重视,而美术工作者们对之倍加爱护、珍惜,也该效法毕加索式的镜子来反射其光华。发扬民间艺术已被提到继承传统的重要方面,其中矿源确乎太丰富了!

工具材料与技法有着紧密的因果关系。宣纸上不宜塑造,托不住浓重多样的彩色。于此作加法难,作减法倒能发挥特殊效果。我在绘画生涯中背负了五十年油彩,基本作加法,近十余年来转到水墨,可说是转向减法,减法还从加法中来。在油彩中,我从繁杂多彩、对比强烈逐步趋向素淡和单纯,投奔于黑与白。这促使我舍油布而迁居宣纸,颇感顺理成章。这是个人的狭隘经验,如果我还是青年,今日才开始用宣纸学习国画,则又将如何对待彩塑呢?思往事,时不再来!我学生时代兼学西画和国画,西画为主,国画为辅。国画老师潘天寿指导从临摹入手,遍临石涛、八大山人、沈周、老莲,上溯元、宋,我一度入了国画系。但感情如野马的年轻人未安于水墨雅淡之乡,后我又跳回了西画系,热恋梵高和马蒂斯去了。时代的变迁,个人的经历和年龄铸造了今天的我,无从后悔,无可自得,自己无法对自己做出客观的评价,倒是可作为别人的借鉴,衰草乃新苗之肥!

欢呼开放,和尚不再固守黄卷青灯,而热衷于云游四方,放眼世界,吸取一切有益的表现手法和工具材料。传统的和外来的济济一堂,互相启示。但诸多技法的创新并不等于艺术的创新,艺术归根还是只能诞生于生活,有所感而发,有所爱而画,画被情催发,几乎忘了技法。作品,是作者与人民感情交流的产儿,在人民中引起共鸣,这交流与共鸣应是艺术之实质,插在这个实质上的柳,多半能成荫吧!
\subsection{笔墨等于零}
脱离了具体画面的孤立的笔墨,其价值等于零。

我国传统绘画大都用笔、墨绘在纸或绢上,笔与墨是表现手法中的主体,因之评画必然涉及笔墨。逐渐,舍本求末,人们往往孤立地评论笔墨,喧宾夺主,笔墨倒反成了作品优劣的标准。

构成画面,其道多矣。点、线、块、面都是造型手段,黑、白、五彩,渲染无穷气氛。为求表达视觉美感及独特情思,作者可用任何手段:不择手段,即择一切手段。果真贴切地表达了作者的内心感受,成为杰作,其画面所使用的任何手段,或曰线、面,或曰笔、墨,或曰××,便都具有点石成金的作用与价值。价值源于手法运用中之整体效益。威尼斯画家委罗内塞(Veronese)指着泥泞的人行道说:我可以用这泥土色调表现一个金发少女。他道出了画面色彩运用之相对性,色彩效果诞生于色与色之间的相互作用。因之,就绘画中的色彩而言,孤立的颜色,赤、橙、黄、绿、青、蓝、紫,无所谓优劣,往往一块孤立的色看来是脏的,但在特定的画面中它却起了无以替代的作用。孤立的色无所谓优劣,则品评孤立的笔墨同样是没有意义的。

屋漏痕因缓慢前进中不断遇到阻力,其线之轨迹显得苍劲坚挺,用这种线表现老梅干枝、悬崖石壁、孤松矮屋之类别有风格,但它替代不了米家云山湿漉漉的点或倪云林的细瘦俏巧的轻盈之线。孰优孰劣?对这些早有定评的手法大概大家都承认是好笔墨。但笔墨只是奴才,它绝对奴役于作者思想情绪的表达。情思在发展,作为奴才的笔墨的手法永远跟着变换形态,无从考虑将呈现何种体态面貌。也许将被咒骂失去了笔墨,其实失去的只是笔墨的旧时形式,真正该反思的应是作品的整体形态及其内涵是否反映了新的时代风貌。

岂止笔墨,各种绘画材料媒体都在演变。但也未必变了就一定新,新就一定好。旧的媒体也往往具备不可被替代的优点,如粗陶、宣纸及笔墨仍永葆青春,但其青春只长驻于它们为之服役的作品的演进中。脱离了具体画面的孤立的笔墨,其价值等于零,正如未塑造形象的泥巴,其价值等于零。
\subsection{邂逅江湖——油画风景与中国山水画合影}
我曾寄养于东、西两家,吃过东家的茶、饭,喝过西家的咖啡、红酒,今思昔,岂肯忘恩负义,先冷静比较两家的得失。

我开始接受的西洋画,从写生入手,观察入手,追求新颖手法,表现真情实感,鄙视程式化的固定技法。刻画人物是主要功课、学艺的必经之途,那是古希腊、罗马、文艺复兴以来的西方传统。历代审美观的递变与发展也都体现在人体表现的递变与发展中。看哪!提香与马蒂斯的人体无从同日而语,但均臻峰巅,各时期辉煌的业绩皆缘于革新精神。是人、是树、是花,在画家眼里都是造型对象,是体现形与色构成的原料或素材,因而很少只局限画风景、肖像或静物的画家,当然这里指的是近代情况。古代的风景画不是从写生中得来,是概念的图像,不感人,真正感人的风景画从印象派开始。

我临摹过大量中国山水画,临摹其程式,讲究所谓笔墨,画面效果永远局限于皴、擦、点、染的规范之内。听老师的话,也硬着头皮临四王山水,如果没有石涛、八大山人、石溪、弘仁等表露真性情的作品,我就不愿学中国山水画了。中国传统山水画以立幅为多,画面由下往上伸展以表现层峦叠嶂,并总结出构图的规律,曰:起、承、转、合。老师说以“起”为最难。“起”是前景,如何设计这第一步,要考虑到文章的全局组合。而这前景总离不开树、石,易于雷同。石涛冲出樊笼,往往突出中景,以最吸引视觉的生动形象组成画面的主体,可说都是写生的形象,是山水的肖像。他公开自己创作的经验与实质:搜尽奇峰打草稿。在油画风景写生中,构图时最难处理的也是前景。因透视现象,前景所占面积最大,但其形象却往往大而空,不如远景重叠多层次,形象丰富而多样。我发现这样一个规律:画面上面积愈大的部分,在整体效果中其作用愈大。因此,如引人入胜的是中景或远景,就要设法挪动占着大面积的前景,或反透视之道而压缩其面积,或移花接木另觅配偶。山水画的“起”由作者精心设计,在油画风景的前景中作者同样煞费心机。这是油画风景与水墨山水邂逅的第一个回合。通过这个回合,我不再局限于一个视点或一个地点写生一幅油画,而移动画架作组合写生,无视印象派的家规。

印象派属最忠实的写实主义,忠实于阴、晴、晨、暮的不同感受。据记载,莫奈画巨幅睡莲时,因要保持其视点不移位,地面挖了沟可让画幅下降入沟以便于挥写画面上部。也重视外师造化的中国山水画家对此当大不以为然,他们行万里路是为了开辟胸中丘壑。偏爱写家乡小景的倪云林,其竹、石、茅亭,疏疏淡淡,则是通向文学意境的小桥流水。

世界是彩色的,印象派再现了彩色世界的斑斓,在流派纷呈的今日艺坛,获得全世界最多观众的恐怕还是印象派作品。印象派认为黑与白不是色彩,由于阳光的作用不可能存在纯黑与纯白。中国山水画主要依靠黑与白,以黑白创造世界,是墨镜中看到的世界。最早的照相是黑白的,彩色世界摄入了黑白照片,人们惊叹其真实。经过了彩色摄影阶段,不少高明的摄影师仍偏爱用黑白。黑白,倒表现了彩色现象的本质。油画风景与水墨山水在彩色与黑白间遭遇了又一个回合。

王国维说,一切写景皆是写情。这适合于油画风景和水墨山水两方面作座右铭。塞尚的风景刀劈斧凿,用色彩建筑坚实的形象,铿锵有声;尤脱利罗利用疏密相间的手法表现哀艳的巴黎,冷冷清清凄凄惨惨戚戚,具东方诗词的情调;梵高的风景地动山摇,强烈的感情震撼了宇宙,鬼哭狼嚎。黄宾虹说西方最上乘的作品只相当于我们的能品,他那一代画家大都对西方绘画一无所知,缘于民族的悲哀。中国优秀的山水画无不重情,但偏向的是文学之情意。自从苏东坡评说王维画中有诗,诗中有画后,诗画间的内在因缘日渐被庸俗化为诗画的互相替代,阻塞了绘画向造型方面的独立展拓。西方是反对绘画隶属于文学的,“文学性绘画”(Peinture,Litteraire)是贬词。我从诗画之乡走出去,对此特别警惕,切忌以文学削弱绘画。这是油画风景与中国山水画遭遇的又一个回合。

情的载体是画面,画面的效果离不开技,没有技,空口说白话。特定的技巧,诞生于特定的创作需要。如不甘心于重复老调,技法永远在更新。黏糊糊的油彩如何表达线的奔放缠绵,她拖泥带水,追不上水墨画及书法的纵横驰骋,她如何利用自身的条件来引进流动的线之表情?水墨画像写字一样,长缨在手,挥毫自如,却也手法有限,对繁花似锦、变化多端的现实世界往往束手无策。因此,懒惰的办法是各自在家吃祖上的老本。但年青一代不甘寂寞,他们闯出家门,闯入世界,油画风景和水墨山水两家的家底被他们翻出来示众了。在西方学习了绘画中的结构规律、平面分割的法则,回头再看自己祖先的杰作,我惊讶地发现:范宽的《谿山行旅图》立足于“方”的基本构成,其效果端庄而厚重;郭熙的《早春图》以“弧”为主调,从树木干枝到群山体态,均一统在曲线的颂歌中,构成恢宏的春之曲;弘仁着墨无多,全凭平面分割之独特手法,表现大自然的宽阔与开合……我曾将中、西方杰出的绘画作品比作哑巴夫妻,虽语言有阻,却深深相爱。若真能达到艺术的至境,油画风景和水墨山水其实是嫡亲姊妹,均系大自然的嫡传。如果说不同的生活习惯和历史背景是中、西文化比较中的复杂课题,则邂逅于同一大自然之前,江河湖海之前,风景画和山水画当一见如故,易于心心相印。中国油画学会倡导举办油画风景和中国山水画的联展,选这个最佳切入口来深入比较中、西画的得失,必将大大推动绘画的民族性与世界性问题的探讨,影响深远。

地球在“缩小”,文化在交融,没有必要,也没有可能固执自己的“纯种”传统,何况传统其实是一连串杂种的继续与发展。印象派认为黑与白不是色彩的论点早被抛弃了,黑、白往往也成了油画的宠儿。莫奈那样固守同一视点的写生方法也只是历史的故事。中国山水画没有西方写生的框框,往往自诩为散点透视,生造“散点透视”这个名称无妨,但实质是臣服于“透视”威望的心态。西方风景从描摹客观物象进入写情、写意,就逐步接近山水画的创作心态,但这绝不意味着他们追上来,我们比他们高明。我们必须深刻认识到山水画在表现手法中的贫乏,面对繁荣、多样、色彩缤纷的现实世界,只靠传统现成的手法来反映人们感受的时代性,必然是一筹莫展。

视觉艺术只靠造型效果,形式美永远是绘画的主要语言、唯一语言,于是大批绘画成了只是形式的游戏,甚至是丑的游戏,形式中的美与丑有点混淆不清了。美感既是可感知的,必具备感情内涵,作者的欢乐、抑郁、孤独、愤世嫉俗的心态必然流露在作品中。在油画风景或中国山水画中都可识别其画中诗、画中情,只是在油画中作者大都专注于美感创造,而山水作者不少是文人,有意寓诗情于画意,所以蔡元培归纳说西洋画近建筑,中国画近文学。我上面提到警惕文学性绘画有损绘画的独立造型美。近几年在世界范围内看多了大量无情无义任性泼洒发泄或不知所云描头画脚的油画,感到很乏味,令人怀念被抛弃或遗忘了的文化底蕴。绘画首先是文化,人看风景,人看山水,各人的视角千变万化,其深层的原因是文化差异。

高居翰先生是研究中国绘画的专家,他看到美国大都会博物馆展出的中国画,整体效果很弱,与西方油画比,吸引不了观众,他站在维护中国画的立场上,感到很怅然。确乎,中国传统绘画除极少数杰作外,挂上墙后显得散漫无力,而表现大自然的山水画却应远看,不只是平铺在案上让人细读细寻去作画中之游。中国画论中虽说“近山取其质,远山取其势”,但当其质其势千篇一律时,画面也就失去了魅力。所以有些西方评论认为中国水墨画已没前途,我不认为这是恶意的当头棒,倒促使我们清醒:我们面临着彻底改革的历史时代。周恩来赞美昆曲《十五贯》,说一个剧本救了一个剧种。我相信今日中国将出现一批有实力的大胆作者和崭新作品来挽救古老而日见衰败的中国画。

在大自然前油画风景和中国山水画一律平等,这里是起跑点,比赛着跑吧,并不划分跑道,路是共有的,路线由各人自由选择,人们只注视谁家的旗帜插上了高峰,关于这旗帜来自东方或西方,则不是问题的关键。
\subsection{曲}
曲,曲折,是文学艺术中不可或缺的结构因素,甚至往往成为作品的主调。称之为歌曲,表明歌唱离不开跌宕、婉转、悠扬,声浪多曲,波状推进,绝非直着嗓子吼叫。在造型艺术中,曲之美丑更为突出。柳腰,柳的姿态之美多半缘于腰部之扭曲,故柳之美尤其显示在早春。早春,刚吐叶芽,枝线飘摇,微风吹来,曲线之美、之媚,最是引人入胜。我曾将之称为披纱垂柳,每年这个季节总要用绘画捕捉“柳如烟”的披纱垂柳,但难尽其妙,常扑空。盛夏,垂柳浓妆,绿荫蔽日,壮实而丰满,但失去了曲线之韵律,所以许多妇女千方百计要减肥。

“曲”构成美?未必,曲背的驼子不美,瘸腿行走也缘于无奈。推敲起来,曲线之美,由于不断变换着运动的方向,扩展了活动的空间。梅兰芳在小小的舞台上走S形,是为了冲破舞台场地的局限;林风眠在方形画幅中用弧曲线竭力营造无尽的宇宙空间。不是部位,不是火候,胡乱扭曲的现象正泛滥于舞台,充斥于画图,人道是东施效颦,不是莺歌燕舞。

中学时代学几何,要备一副三角板和两块曲线板,作为画直线和曲线的依赖。绘画中离不开曲线和直线,但无须曲线板和三角板,因那曲线和直线并非绝对意义上的曲与直,而彼此间有着相互渗透的微妙关系。人的脊椎曲而直,似乎是一切有关曲直美丑感的尺度。书法写一横,一波三折,绘画中的直线有时仿屋漏之痕,屈曲延伸。柔中有刚,曲线缠绵中也隐隐受控于横直线的秩序,否则易流于油滑。大师、工匠、民间艺人,在创造美感中对把握曲直之间的分寸,心有灵犀一点通,这个分寸啊,确乎成为美丑的分界线。

古代地大人稀,建造宫殿或大宅以占地面积之大摆威风,吓人。因无法建高层,便着力于厚实,城墙、宫门、房顶都有千年万世砸不烂的厚重、恒久感,于此形成了中国古建筑那种矮墩墩匍匐于大地的独特风格。寓美于朴,屋脊、飞檐翘起了曲线。北方庙宇崇实,飞檐只微微略翘,曲有度;南方亭榭空灵,飞檐高高翘起,长长的曲线似欲腾飞,工匠们把握了美丑之间的曲直分寸。今日全世界大都市高楼林立,几乎没有留给曲线扭动回旋的余地了。巴黎的塞纳河,伦敦的泰晤士河,在弯曲着穿行闹市,似乎给直线纵横的都市雕凿了美丽的大曲线,凡有这种大河的首都也可说是一种骄傲。

“孤松矮屋老夫家”,古代房矮,那高高的孤松,有风骨,有曲直之美,构成了画境。今日的大城市,难觅孤松矮屋之家,老夫们也都住入了高楼,要赏孤松,必须下楼,高楼矮松住宅区,着实委屈了高傲的松。驱车过闹市,偶见杂树成丛,那是最美最美的城中风景了。在石林似的新建筑群中被保留住的老树,即使瘦骨嶙峋,但它前俯后仰、曲曲弯弯的体态,展现了曲线之魅力,真是城中珍异。直线统治的城市呼唤曲线,美丽的人生曲线!
\subsection{魂寓何处——美术中的民族气息杂谈}
“中国水墨画已没有前途”,这是我听到过的一些西方画家对中国画的看法。他们并非出于恶意,这观点却令我警惕。杞人忧天,我本来也一向感到陈陈相因、千篇一律的中国水墨画已日暮途穷,终将枯死于世界艺术的百花园中。中华民族几千年的辉煌绘画岂能断子绝孙,华夏子孙皆曰:绝不可能。周恩来赞美昆曲《十五贯》,说一个剧本救了一个剧种。今日需要,也必将有一批崭新面貌的水墨画来展拓传统,开辟新径。展拓与开辟包含着突破,突破陈旧,被突破的方面不可避免会感到痛楚。

“数典忘祖”“西方中心”等等忧虑出于维护民族特色的善良愿望、忠贞立场。潘天寿说过中、西画要拉开距离,本意是发挥民族特色,但也被人利用作为排斥西方,反对中、西结合,只求纯种的借口。吾爱吾师,吾更爱真理,潘老师自己艺术创作的体验涵盖不了民族艺术的整体发展,何况他本人的杰出成果恰恰与西方现代绘画的探索在某些层次上碰了面(见拙作《高山仰止》)。认为中国水墨画没有前途,当指其内涵之日益单调与形象之不断重复,而这样的画图正泛滥于国内国外。如今很多中青年画家正在努力丰富画面的内涵与形式,宣纸上的用武之地果真比不上油布领域宽阔?我自己的体会是既不服气,也确有隐忧,我们面临着不革新便衰亡的抉择。

写生的过程是从恋爱到结婚的整体过程。不再写生了,只凭照片嫁接,甚至只是照片的抄袭,这类似没有恋爱过程的婚姻,这样的婚姻今日比比皆是。中西结合属异国婚姻,其美满者亦必缘于爱情。无爱的婚姻有诸多目的,各样目的也都渗入了艺术制作中。二十世纪二三十年代时有的画家对洋人显示其中国画,对国人又炫耀其油画,欺蒙观众的无知。今天大量的中国人用油彩和画布作画了,中国绝不可能成为西洋艺术的殖民地;西洋人也不乏试用宣纸、毛笔之类的中国传统工具作画,他们绝非想当我们祖先的孝子贤孙。作画材料和技法非专利,谁也无权霸占,艺术的民族特色又在何处显示、隐藏、潜伏?

我童年觉得洋人都很丑,后来能区分洋人中大有美丑之别,美丑呈现在个体中,正如我们中国人自己也大有美丑之差异。艺术作品的美丑只能从每件作品自身来剖析,难于以整个民族或地区来概括。民族的、地区的特色显然是由于长期历史的、地理的局限等等生活环境所形成。德拉克洛瓦画的《但丁之舟》是西洋画,如果我们用他那手法表现了汨罗江上的屈原,该如何区别其民族性呢?郎世宁的翎毛、走兽,李毅士的《长恨歌》,这类作品都凭采用中国题材赋予中国风貌,而从其造型艺术本身的语言分析,却全无新意,而且也算不上高水平的西洋画技巧,但他们尝试了中、西结合的可能性、必然性。达·芬奇的素描山水与黄公望的《富春山居图卷》颇为相似;波提切利的作品突出线造型、平面感、衣带飘摇感,大异于拉斐尔、提香等浑圆丰厚的立体氛围,独具东方情致。尤脱利罗的作品中可感到冷冷清清凄凄惨惨,以及“小楼一夜听春雨,深巷明朝卖杏花”的中国诗情,加之他表现手法中强调平面分割的对照及线之效果,我最早喜爱其作品也许缘于吻合了我的中国品位,而中国传统的或民间的绘画中也同样可发现西方所探索的因素。近代除潘天寿外,虚谷作品中的几何构成及点、线、团块间的对照与联系也与西方现代绘画的形式感异曲同工。我有一次在印尼看到一个硕大无朋的胖妇人,那是马约尔与毕加索等所追求的量感美典型,我也觉得是发挥量感美极好的对象,回来为之作了一幅油画。一见此画中肥婆的友人都感到惊异,我说我画的是洋阿福,于是友人们立即回复了平常心态。无锡的胖阿福已被多少代中国人欣赏,洋人也会对之青睐。

我一向着眼于中、西方审美之共性。我爱传统绘画之美,并曾大量临摹,深切地爱过,仍爱着。我也真正爱西方绘画之美,东也爱,西也爱,爱不专一,实缘真情,非水性杨花也。正相反,倒是人们如何会只爱东方或只崇西方呢?审美中也有大量的偏食者!我们的民族近代长期受侵略,遭歧视,自卑激发了自尊。我年轻时代留学异国,在被歧视的环境中我是带着敌情观念学习的,并深感我们民族几千年的艺术成熟独立于世界艺坛而无愧。但也正由于学习了西方之优,在比较中更认清自己民族的特色、不足和欠缺。我们民族传统之博大是由于历史悠久,积累深厚。积累者,不断吸收与创造之谓也,其对立面是孤陋寡闻。

二十世纪六十年代我用油画写生江南,白墙黑瓦、桃柳交错、春阴漠漠,绝非西方油画中的风物与情调了,我请李可染看,谅来当是知音,因他不久前到富春江等地用水墨写生的一批风景与我的油画写生具相似的追求。当时有人批评他,不认他的写生属传统中国画,当然更有人不认我的作品是正规油画,苏联专家先就认为江南风景不宜于作油画,苏联没有写出“杏花春雨江南”的诗人。有些中国画家定居在西方了,各有乡土情怀,或以西方技法表现中国题材,或以中国“气韵生动”的品位融入西方的抽象表现,或以民间工艺的审美观结合了西方现代的夸张与狂放……如果出于爱情的自然结合,诞生的混血儿多半留有某种或隐或现的胎记,慧眼人易于识别自己民族的印痕。“领异标新二月花”,在芸芸艺坛上扬名实非易事,于是为标新而装腔作势,故弄玄虚,焚香祈求老、庄、佛、禅……处处暴露无爱婚姻的无奈。“前卫”(avant garde)一词,本身无褒贬之意,其含义等同于创新,而其间真情探索与欺世盗名则不可同日而语。

绝无永远的纯种,周口店祖先的头骨留下了珍贵的文物,但已不是我们子孙的脸型,儿子未必像父亲,不必像父亲。遗传基因是科学家的课题,艺术中的遗传基因更为隐蔽,她往往只体现在感受中。即使面貌不像了,也许基因却呈现在脉搏里,脉搏里有节奏,或者节奏寄寓于脉搏,说得玄乎点,有魂,魂寓何处?寓于历代文化的背景中,寓于各自的苦难与悲愤中,寓于扬眉吐气中……而材料将不是分类的标志,文房四宝将不是国画之唯一基石,传统的优秀笔墨已凝固在传统的品位中。在世界百花园中,奇花异草引人瞩目,奇花异草必有其独特的土壤与根源,顺藤摸瓜,最终能摸到其故国的、民族的因素。但如根本缺乏对艺术的纯正爱恋,一味标榜民族的特色,强加于人,以自尊掩饰自卑,只能掀起假花市场。
\subsection{说“变形”}
\end{document}
